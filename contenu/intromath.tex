\documentclass[../main.tex]{subfiles}
\begin{document}

\section{Intro mathématique}
\renewcommand{\arraystretch}{2} % Augmente l'espacement des lignes de 50%
  % \subsection{Logarithmes}
  % \[
  %   y=\log_a(x) \Leftrightarrow a^y = x
  % \]
  % \[
  %   \log_a(xy) = \log_a(x) + \log_a(y)
  % \]
  % \[
  %   \log_a\left(\frac{x}{y}\right) = \log_a(x) - \log_a(y)
  % \]
  % \[
  %   \log_a\left(\frac{1}{x}\right) = -\log_a(x)
  % \]
  % \[
  %   \log_a\left(x^p\right) = p\log_a(x)
  % \]
  % \[
  %   y=\log(x) \Leftrightarrow 10^y = x 
  % \]
  % \[
  %   y=\ln(x) \Leftrightarrow e^y = x 
  % \]
  %
  % Changement de base : 
  % \[
  %   \log_a(x)=\frac{\log_b(x)}{\log_b(a)}=\frac{\log(x)}{\log(a)}=\frac{\ln(x)}{\ln(a)}
  % \]

\subsection{Trigo}
\begin{center}
  \begin{tabular}{cc}
    \trigod{\cos^2(\alpha) + \sin^2(\alpha) = 1}{ \tan(\alpha)=\frac{\sin(\alpha)}{\cos(\alpha)}=\frac{1}{\cot(\alpha)}}
    \trigod{ \frac{1}{\cos^2(\alpha)}=1+\tan^2(\alpha)}{ \frac{1}{\sin^2(\alpha)}=1+\cot^2(\alpha)}
  \end{tabular}
  \vspace{0.1cm}\hrule\vspace{0.1cm}
  \scalebox{0.75}{
    \begin{tabular}{ccc}
        % \toprule
      \trigo{\Cos{\alpha+2\pi}=\Cos{\alpha}}{\Sin{\alpha+2\pi}=\Sin{\alpha}}{\Tan{\alpha+\pi}=\Tan{\alpha}}
        % \midrule
      \trigo{\Cos{-\alpha}=\Cos{\alpha}}{\Sin{-\alpha}=-\Sin{\alpha}}{\Tan{-\alpha}=-\Tan{\alpha}}
      \trigo{\Cos{\pi-\alpha}=-\Cos{\alpha}}{\Sin{\pi-\alpha}=\Sin{\alpha}}{\Tan{\pi-\alpha}=-\Tan{\alpha}}
      \trigo{\Cos{\pi+\alpha}=-\Cos{\alpha}}{\Sin{\pi+\alpha}=-\Sin{\alpha}}{\Tan{\pi+\alpha}=\Tan{\alpha}}
      \trigo{\Cos{\frac{\pi}{2}-\alpha}=\Sin{\alpha}}{\Sin{\frac{\pi}{2}-\alpha}=\Cos{\alpha}}{\Tan{\frac{\pi}{2}-\alpha}=\Cot{\alpha}}
      \trigo{\Cos{\frac{\pi}{2}+\alpha}=-\Sin{\alpha}}{\Sin{\frac{\pi}{2}+\alpha}=\Cos{\alpha}}{\Tan{\frac{\pi}{2}+\alpha}=-\Cot{\alpha}}
        % \bottomrule
    \end{tabular}
  }
  \vspace{0.1cm}\hrule\vspace{0.1cm}
  \scalebox{0.6}{
    \begin{tabular}{cc}
        % \toprule
      \trigod{\Cos{\alpha+\beta}=\Cos{\alpha}\Cos{\beta}-\Sin{\alpha}\Sin{\beta}}{\Cos{\alpha-\beta}=\Cos{\alpha}\Cos{\beta}+\Sin{\alpha}\Sin{\beta}}
      \trigod{\Sin{\alpha+\beta}=\Sin{\alpha}\Cos{\beta}+\Cos{\alpha}\Sin{\beta}}{\Sin{\alpha-\beta}=\Sin{\alpha}\Cos{\beta}-\Cos{\alpha}\Sin{\beta}}
      \trigod{\Tan{\alpha+\beta}=\frac{\Tan{\alpha}+\Tan{\beta}}{1-\Tan{\alpha}\Tan{\beta}}}{\Tan{\alpha-\beta}=\frac{\Tan{\alpha}-\Tan{\beta}}{1+\Tan{\alpha}\Tan{\beta}}}
        % \bottomrule
    \end{tabular}
  }
  \vspace{0.1cm}\hrule\vspace{0.1cm}
  \scalebox{0.9}{
    \begin{tabular}{l}
        % \toprule
      \(\displaystyle\Cos{2\alpha}=\Coss{\alpha}-\Sins{\alpha}=1-2\Sins{\alpha}=2\Coss{\alpha}-1\)\\
      \(\displaystyle\Sin{2\alpha}=2\Sin{\alpha}\Cos{\alpha}\)\\
      \(\displaystyle\Tan{2\alpha}=\frac{2\Tan{\alpha}}{1-\Tans{\alpha}}\)\\
        % \bottomrule
    \end{tabular}
  }
  \vspace{0.1cm}\hrule\vspace{0.1cm}
  \scalebox{0.8}{
    \begin{tabular}{cc}
        % \toprule
      \trigod{\Coss{\frac{\alpha}{2}}=\frac{1+\Cos{\alpha}}{2}}{\Sins{\frac{\alpha}{2}}=\frac{1-\Cos{\alpha}}{2}}
      \trigod{\Tans{\frac{\alpha}{2}}=\frac{1-\Cos{\alpha}}{1+\Cos{\alpha}}}{\Tan{\frac{\alpha}{2}}=\frac{1-\Cos{\alpha}}{\Sin{\alpha}}=\frac{\Sin{\alpha}}{1+\Cos{\alpha}}}
        % \bottomrule
    \end{tabular}
  }
  \vspace{0.1cm}\hrule\vspace{0.1cm}
  \scalebox{0.6}{
    \begin{tabular}{cc}
      \trigod{
        \Cos{\alpha}+\Cos{\beta}=2\Cos{\frac{\alpha+\beta}{2}}\Cos{\frac{\alpha-\beta}{2}}
      }{
        \Cos{\alpha}-\Cos{\beta}=-2\Sin{\frac{\alpha+\beta}{2}}\Sin{\frac{\alpha-\beta}{2}}
      }
      \trigod{
        \Sin{\alpha}+\Sin{\beta}=2\Sin{\frac{\alpha+\beta}{2}}\Cos{\frac{\alpha-\beta}{2}}
      }{
        \Sin{\alpha}-\Sin{\beta}=2\Cos{\frac{\alpha+\beta}{2}}\Sin{\frac{\alpha-\beta}{2}}
      }
      \trigod{
        \Tan{\alpha}+\Tan{\beta}= \frac{\Sin{\alpha+\beta}}{\Cos{\alpha}\Cos{\beta}}
      }{
        \Tan{\alpha}-\Tan{\beta}= \frac{\Sin{\alpha-\beta}}{\Cos{\alpha}\Cos{\beta}}
      }
    \end{tabular}
  }
  \vspace{0.1cm}\hrule\vspace{0.1cm}
\end{center}
\[
  a\Cos{\alpha} +b\Sin{\alpha} = \sqrt{a^2 + b^2}\Cos{\alpha - \varphi}
\]
\[
  \varphi=\Arccos{\frac{a}{\sqrt{a^2 + b^2}}}=\Arcsin{\frac{b}{\sqrt{a^2 + b^2}}}=\Arctan{\frac{b}{a}}
\]
  % \begin{center} \vspace{0.1cm}\hrule\vspace{0.1cm} \end{center}
  % \begin{align*}
  %   2\Cos{\alpha}\Cos{\beta}&=\Cos{\alpha+\beta} + \Cos{\alpha-\beta}\\
  %   2\Cos{\alpha}\Sin{\beta}&=\Sin{\alpha+\beta} - \Sin{\alpha-\beta}\\
  %   2\Sin{\alpha}\Sin{\beta}&=-\Cos{\alpha+\beta} + \Cos{\alpha-\beta}\\
  % \end{align*}
  % \begin{center} \hrule\vspace{0.1cm} \end{center}
\[
  \cos(x)=a \Rightarrow
  \begin{cases}
    x = \arccos(a) + k\cdot 2\pi \,\,\,\,\text{ou}\\
    x = - \arcsin(a) + k\cdot 2\pi 
  \end{cases}
\]

\[
  \sin(x)=a \Rightarrow
  \begin{cases}
    x = \arcsin(a) + k\cdot 2\pi \,\,\,\,\text{ou}\\
    x = \pi - \arcsin(a) + k\cdot 2\pi 
  \end{cases}
\]

\[\tan(x)= a \Rightarrow x = \arctan(a) + k\cdot\pi\]
  % \begin{center} \vspace{0.1cm}\hrule\vspace{0.1cm} \end{center}
\[
  a^2 = b^2 + c^2 - 2bc\Cos{\alpha}\phantom{text} \frac{a}{\sin(\alpha)}=\frac{b}{\sin(\beta)} 
\]


\renewcommand{\arraystretch}{1}

\subsection{Géométrie}
\subsubsection{Vecteur}
\req{1}{
  \ac\cdot\vv{b}=||\ac||\,\,||\vv{b}||\cos\varphi \phantom{textus} ||\ac||=\sqrt{\ac\cdot\ac} 
}

Projection de \(\vv{b}\) sur \(\ac\) : 
\[
  \vv{b}' = \frac{\ac\cdot\vv{b}}{||\ac||^2}\ac \phantom{texttext} 
  ||\ac\times\vv{b}||=||\ac||\,\,||\vv{b}||\sin\varphi
\]
\[
  \ac\times\vv{b}=\begin{pmatrix}
    a_2b_3 - a_3b_2 \\ 
    a_3b_1 - a_1b_3 \\ 
    a_1b_2 - a_2b_1
  \end{pmatrix}
  =-(\vecb\times\veca)
\]
\[
  \veca\times(\vecb\times\vecc)=(\veca\cdot\vecc)\vecb - (\veca\cdot\vecb)\vecc
\]

\subsection{Dérivés et integrales}
  % \subsubsection{Dérivation}
  % \(\displaystyle
  %   f'(x)=\diff{f}{x} \phantom{text} f''(x)=\diff[2]{f}{x} \phantom{text} f^{(n)}(x)=\diff[n]{f}{x}
  % \)
  %
  % \(\displaystyle
  %   (f+g)'(x)=f'(x)+g'(x) \phantom{text} (\lambda\cdot f)'(x)=\lambda f'(x)
  % \)
  %
  % \(\displaystyle
  %   (f\cdot g)'(x)=f'(x)g(x)+f(x)g'(x)
  % \)
  %
  % \(\displaystyle
  %   \left(\frac{f}{g}\right)'(x)=\frac{f'(x)g(x)-f(x)g'(x)}{g^2(x)}
  % \)
  %
  % \(\displaystyle
  %   (g\circ f)'(x)=g'(f(x))f'(x)
  % \)
  % \subsubsection{Intégration}
  % \begin{tabular}{p{1.8cm}l}
  %   \multirow{2}{*}{Linéarité} & \eqentry{ \int(f(x) + g(x))\dx = \int f(x)\dx + \int g(x)\dx } \\ 
  %
  %                              & \eqentry{ \int \lambda f(x)\dx = \lambda\int f(x)\dx } \\
  %
  %   Par parties & \eqentry{ \int f'(x)g(x)\dx = f(x)g(x) - \int f(x)g'(x)\dx} \\
  %
  %   Par substitution & \eqentry{ \int g(f(x))f'(x)\dx = G(x) + c} \\ 
  %
  %   Par changement de variable (\(x=f(t)\)) & \eqentry{ \int g(x)\dx = \int g(f(t))f'(t) \dt } \\
  % \end{tabular}

  % Linéarité : 
  %
  % \(\displaystyle \int(f(x) + g(x))\dx = \int f(x)\dx + \int g(x)\dx \)
  %
  % \(\displaystyle \int \lambda f(x)\dx = \lambda\int f(x)\dx \)

Par parties : 

\(\displaystyle \int f'(x)g(x)\dx = f(x)g(x) - \int f(x)g'(x)\dx \)

Par substitution : 
\(\displaystyle \int g(f(x))f'(x)\dx = G(x) + c \)

Par changement de variable (\(x=f(t)\)) : 

\(\displaystyle \int g(x)\dx = \int g(f(t))f'(t) \dt \)

\renewcommand{\arraystretch}{2}
\begin{center}
  \scalebox{0.6}{
    \begin{tabular}{c|c|c}
      \toprule
      \eqligne{f(x)}{f'(x)}{F(x)}
      \midrule
      \eqligne{a}{0}{ax}
      \eqligne{x}{1}{x^2}
      \eqligne{\frac{1}{x}}{-\frac{1}{x^2}}{\ln|x|}
      \eqligne{\sqrt{x}}{\frac{1}{2\sqrt{x}}}{\tfrac{2}{3}x\sqrt{x}}
      \eqligne{\frac{1}{\sqrt{x}}}{-\tfrac{1}{2}\sqrt{x^3}}{2\sqrt{x}}
      \eqligne{x^n}{nx^{n-1}}{\frac{1}{n+1}x^{n+1}}
      \eqligne{\frac{1}{x^{n}}}{\frac{-n}{x^{n-1}}}{\frac{-1}{n-1}\frac{1}{x^{n-1}}}
      \eqligne{\sqrt[n]{x}}{\frac{1}{n}\sqrt[n]{x^{n-1}}}{\frac{n}{n+1}\sqrt[n]{x^{n+1}}}
        % \midrule
      \eqligne{e^{ax}}{ae^{ax}}{\frac{1}{a}e^{ax}}
      \eqligne{b^{ax}}{ab^{ax}\ln(b)}{\frac{b^{ax}}{a\ln(b)}}
      \eqligne{\ln(x)}{\frac{1}{x}}{x(\ln(x) - 1)}
      \eqligne{\log_{a}(bx)}{\frac{1}{x\ln(a)}}{x\log_{a}\left(\frac{bx}{e}\right)}
        % \midrule
      \eqligne{\sin(x)}{\cos(x)}{-\cos(x)}
      \eqligne{\cos(x)}{-\sin(x)}{\sin(x)}
      \eqligne{\tan(x)}{\frac{1}{\cos^2(x)}}{-\ln|\cos(x)|}
        % \eqligne{\sin^2(x)}{2\sin(x)\cos(x)}{\tfrac{1}{2}(x-\sin(x)\cos(x))}
        % \eqligne{\cos^2(x)}{-2\sin(x)\cos(x)}{\tfrac{1}{2}(x+\sin(x)\cos(x))}
        % \eqligne{\tan^2(x)}{2 \frac{\tan(x)}{\cos^2(x)}}{\tan(x) - x}
        % \midrule 
      \eqligne{\arcsin(x)}{\frac{1}{\sqrt{1-x^2}}}{x\arcsin(x) + \sqrt{1-x^2}}
      \eqligne{\arccos(x)}{-\frac{1}{\sqrt{1-x^2}}}{x\arccos(x) - \sqrt{1-x^2}}
      \eqligne{\arctan(x)}{\frac{1}{1+x^2}}{x\arctan(x) - \tfrac{1}{2}\ln(1+x^2)}
        % \midrule
      \eqligne{g(x)g'(x)}{g'(x)^2 +g(x)g''(x)}{\tfrac{1}{2}g(x)^2}
      \eqligne{\frac{g'(x)}{g(x)}}{\frac{g''(x)g(x)-g'(x)^2}{g(x)^2}}{\ln|g(x)|}
      \bottomrule
    \end{tabular}
  }
\end{center}
\renewcommand{\arraystretch}{1} % Augmente l'espacement des lignes de 50%
\subsection{Polynômes Taylor}
\[
  P_N(x)=\eval{\sum_{n=0}^{N}\frac{1}{n!}\diff[n]{f}{x}}_{x=x_0}(x-x_0)^n
\]

\end{document}
