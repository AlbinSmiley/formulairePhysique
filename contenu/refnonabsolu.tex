\documentclass[../main.tex]{subfiles}

\begin{document}

\section{Referentiel non-absolu}
\[
  \vian=\tfrac{1}{2}\sum_{i=1}^{3}\left(\vv{u}_i'\times\dv{\vv{u}_i'}{t}\right) \phantom{thisis} \dv{\vv{u}_i'}{t}=\vian\times\vv{u}_i'
\]
\[
  \vi_p=\vi_p'+\vi_{O'} + \vian\times\vv{O'P}
\]
\[
  \ac_p = \ac_p' + \overbrace{2\vian\times\vi_p'}^{\text{Coriolis}} + \underbrace{\ac_{O'} + \overbrace{\vian\times(\vian\times\vv{O'P})}^{\text{Centripète}} + \dv{\vian}{t}\times\vv{O'P}}_{\text{Entrainement}}
\]
\[
  m\ac_p'=\sum\F - \underbrace{\F_i}_{m\ac_i}
\]

\subsection{Moment cinétique}
\[
  \vv{L}_0=\vv{OP}\times\vv{p}
\]

Théorème du moment cinétique :
\[
  \dv{\vv{L}_0}{t}=\vv{OP}\times\F=\vv{M}_0
\]

  % \begin{align*}
  %   \vv{L}_A=\vv{L}_O + m\left(\vv{AO}\times\vi\right)\dv{}{t}||\mc_O||^2 \\
  %   =2\mc_O\cdot\dv{}{t}\mc_O = 2\mc_O\cdot\mf^{ext}_O \\ 
  %   = 0 \Rightarrow \mc_O \text{ Conservé }
  % \end{align*}

\subsection{Mouvement sous l'action d'une force centrale}
Energie potentielle effective : \(V_{eff}=U+\frac{\mc_O^2}{2mr^2}\)

  % \section{Solide 2D}
  % Dans le cas générale, O le point du centre de rotation : 
  % \[
  %   \mc_{tot,O}=\sum_i\mc_{m_i}=\sum_i^n I_i\Omega_i + \sum_j^m m_j\vv{OM}_j\times\vi_j
  % \]
  %
  % Avec \(n\) corps solides et \(m\) points materiels. 
  %
  % \subsection{Rotation autour d'un axe de symétrie}
  % \( \mc_G = I_{\Delta G}\vian \) tout le temps valable 
  % \[
  %   K = \tfrac{1}{2}Mv_G^2 + \tfrac{1}{2}I_{\Delta G}\omega^2
  % \]
  % \(\mc_G\) est parallèle à \(\omega\), \(I_{\Delta G} = \sum_im_i||\vv{GP}_{i\bot}||^2\)
  %
  % \subsection{Rotation autour d'un axe instantané fixe \(\Delta\) parallèle à un axe de symétrie}
  % \[\mc_O=I_{\Delta}\vian\]
  % \[
  %   I_{\Delta} = Md^2 + I_{\Delta G} 
  % \]
  % \[
  %   \mc_G = \mc_O + \vv{GO}\times m\vi_O
  % \]
  % \[
  %   K = \tfrac{1}{2}I_{\Delta}\omega^2
  % \]
  % \[\mc_O||\vian\]

\end{document}
