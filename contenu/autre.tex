\documentclass[../main.tex]{subfiles}

\begin{document}

\section{Kepler}
\[
  \dv{\mathcal{A}}{t}=\frac{||\vv{L}_0||}{2m}=\text{const} \phantom{text} \frac{a_A^3}{T_A^2}=\frac{a_B^3}{T_B^2}\phantom{text} \frac{T^2}{R^2}=\frac{4 \pi^2}{G(M_s + M)}
\]
\(a_A\approx R_A\) pour un cercle
\section{Chocs}
\[
  \Qm^{in}=\Qm^{fin} \phantom{texttext}
  \mc^{in}_O=\mc^{fin}_O
\]
\[ \Delta K \overset{?}{=}
  \begin{cases}
    0 = \implies \text{choc élastique}\\
    0 \neq \implies \text{choc inélastique}
  \end{cases}
\]

Il y a \textbf{choc mou} lorsque les deux masses se collent. La conservation de la quantité de mouvement est vérifié mais \(\Delta K \neq 0\).

\subsection{Choc 1-D, élastique}
\[
  \begin{cases}
    m_1 v_{1i} + m_2v_{2i} = m_1 v_{1f} + m_2 v_{2f} \\ 
    \tfrac{1}{2}m_1 v_{1i}^2 + \tfrac{1}{2}m_1 v_{2f}^2 - \tfrac{1}{2}m_1 v_{1f}^2 = 0
  \end{cases}
\]

Ce système a pour solutions :
\[
  \begin{cases}
    v_{1f} = \frac{(m_1-m_2)v_{1i} + 2m_2v_{2i}}{m_1+m_2} \\
    v_{2f} = \frac{(m_2-m_1)v_{2i} + 2m_1v_{1i}}{m_1+m_2}
  \end{cases}
\]

\subsection{Choc inélastique}
\[
  \Delta K 
  \begin{cases}
    > 0 \implies \text{exo-énergetique} \\
    < 0 \implies \text{endo-énergetique}
  \end{cases}
\]

\subsection{Choc mou (parfaitement inélastique)}
Les deux objet se collent (\(\vi_{1f}=\vi_{2f}=\vi_{\text{mou}}\)). Dans le cas particulier ou \(\vi_{2i}=0\), la conservation de la quantité de mouvement nous dit que : 
\[
  m_1\vi_{1i} = m_1\vi_{1f} + m_2\vi_{2f} = (m_1 + m_2)\vi_{\text{mou}} 
\]
\[
  \vi_{\text{mou}} = \frac{m_1}{m_1 + m_2}\vi_{1i}
\]
\[
  \Delta K = K^{\text{fin}} - K^{\text{in}}= -\tfrac{1}{2}\frac{m_1m_2}{m_1+m_2}v_{\text{mou}}^2 < 0 
\]

\section{Système à masses variable}
\[ M(t)\dv{\vi}{t}=\sum\F^{\text{ext}} + \vi_{\text{rel}}\dv{M}{t} \]

\section{Moment d'inertie}
\(\Delta_C=\)axe du cylindre

\scalebox{0.7}{
  \begin{tabular}{p{3cm}l}
    Boule (pleine) & \(\displaystyle I = \frac{2}{5}mR^2\)\\

    Sphère (creuse) & \(\displaystyle I = \frac{2}{3}mR^2\)\\

    Anneau (\(\Delta_c\)) & \(\displaystyle I = mR^2\)\\

    Anneau (\(\bot\Delta_c\)) & \(\displaystyle I = \frac{1}{2}mR^2\)\\

    Anneau (épaisseur \(\neq 0\), \(\Delta_c\)) & \(\displaystyle I = \frac{1}{2}m(R_1^2 + R_2^2)\)\\

    Cylindre (\(\Delta_c\)) & \(\displaystyle I = \frac{1}{2}mR^2\)\\

    Cylindre (\(\bot\Delta_c\)) & \(\displaystyle I = \frac{1}{4}mR^2 + \frac{1}{12}mL^2\)
  \end{tabular}
}
\end{document}
